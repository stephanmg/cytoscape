\section{Introduction}
\begin{frame}{Introduction I}
\dots why should we care about glycolysis and gluconeogeneses
\end{frame}

\begin{frame}{Introduction}
	\begin{itemize}
		\item using the tool CellDesigner to build a glycolysis/gluconeogenese network,
		\item import it into the tool Cytoscape,
		\item analyse the imported network and
		\item implement missing centralities and indices by coding some plugins
	\end{itemize}
\end{frame}

\begin{frame}{Glycolysis/Gluconeogenese}
	\begin{itemize}
		\item catabolic linear pathway of glycolysis deals with the breakdown and extraction of energy from glucose
		\item the reverse anabolic process - gluconeogenesis - is equally important
		\item gluconeogenesis helps to keep blood glucose levels within critical limits
		\item these processes provide many points for regulation
	\end{itemize}
\end{frame}

\begin{frame}{Glycolysis}
	\begin{figure}[htbp]
	   \centering
	   \includegraphics[width=0.6\textwidth]{inc/img/circle}
	   \caption{Glycolysis vs. Gluconeogenese}
	   \label{fig:circle}
	\end{figure}
\end{frame}

\begin{frame}{Glycolysis}
	\begin{figure}[htbp]
	   \centering
	   \includegraphics[width=0.6\textwidth]{inc/img/equation}
	   \caption{Glycolysis vs. Gluconeogenese}
	   \label{fig:circle}
	\end{figure}
\end{frame}

\begin{frame}{Glycolysis and Disease}
	\begin{itemize}
		\item Genetic disease - mutations are generally rare due to importance of the metabolic pathway
		\item Cancer - typically tumor cells have glycolytic rates that are up to 200 times higher
		\item Other disease - disfunctioning glycolysis or glucose metabolism has been associated with some other diseases
	\end{itemize}
\end{frame}

\begin{frame}{Glycolysis and Cancer}
	\begin{itemize}
		\item Ubiquitous gene overexpression appears to be restricted to glycolysis, in conclusion, Glycolysis is indeed special.
		\item this may also be of some interest for therapy, increased Glucose consumption can be observed with clinical tumour imaging 
		\item Gene expression patterns in general can be modified by external factors such as drugs or components of nutrition
		\item one may envision substances that modify expression of glycolysis genes as complementary to conventional cancer therapies
	\end{itemize}
\end{frame}

\begin{frame}
	\begin{itemize}
		\item
	\end{itemize}
\end{frame}

\begin{frame}
	\begin{itemize}
		\item
	\end{itemize}
\end{frame}

\begin{frame}{Literature}
	\begin{itemize}
		\item R. A. Gatenby and R. J. Gillies, Why do cancers have high aerobic glycolysis?, Nature Review Cance, Vol. 4, p. 891-899, November 2004.
		\item B. Altenberga and K.O. Greulich, Genes of glycolysis are ubiquitously overexpressed in 24 cancer classes, Genomics Vol. 84, p. 1014-1020, September 2004.
	\end{itemize}
\end{frame}

\begin{frame}{Cell Designer}
	\begin{itemize}
		\item tried to model a glycolysis network
		\item tool very user unfriendly
		\item export into a format for cytoscape resulted in "strange" networks
		\item this task was skipped
	\end{itemize}
\end{frame}

\begin{frame}{Cytoscape}
	\begin{itemize}
		\item provides many useful plugins 
		\item plugin KGMLReader used to import KEGG Glycolysis network
		\item modified the network to make it user readable
		\item 
	\end{itemize}
\end{frame}


\begin{frame}{Introduction II}
\dots basic setup of our two networks ... how we modelled them and what did not work (CellDesigner) ...
\end{frame}


